\chapter{Preliminaries}

%%%%%%%%%%%%%%%%%%%%%%%%%%%%%%%%
\section{A Brief History of *GL}
\summary{Gives an account of WebGL's ancestry (OpenGL, OpenGL ES), motivation, and rapid growth.  Also briefly mentions impediments at the time of writing, such as security concerns and IE support.}

%%%%%%%%%%%%%%%%%%%%%%%%%%%%%%%%%%%%%%%%%%%%%
\section{Building Giza: Literate Programming}
\summary{Describes our coding conventions and the \code{giza} library that is developed over the course of the book.}

%%%%%%%%%%%%%%%%%%%%%%%%%%%%%%%%%%%%
\section{The Assembly Line Metaphor}
\summary{High-level overview of the WebGL rendering pipeline.}

%%%%%%%%%%%%%%%%%%%%%%%%%%%%
\section{The Canvas Element}
\summary{Explains the \code{width} and \code{height} attributes and how to handle high-dpi displays (e.g., Apple Retina).}

The \code{\index{canvas}} element is one of the cornerstones of the HTML5 platform.  It provides rich drawing APIs for both 2D and 3D graphics.  The 2D API will not be discussed much in this book (except briefly in Chapter 8), and the 3D API is, of course, WebGL.

\subsection{Dealing with Size and High-Resolution Screens}

By default, \code{<canvas>} is a block level element, similar to \code{<div>}.  This means that it can only be a child of \code{<body>}, and that it's typically rendered with line breaks before and afterwards.

An important distinction between canvas and other HTML elements is its two sets of \index{size}s.  One size is the \emph{display area}, computed with familiar \index{CSS} mechanisms.  The other size is the \emph{content size}, specified with explicit \code{\index{width}} and \code{\index{height}} attributes.  The content size specifies an off-screen drawing surface that gets scaled into the display area on the web page.

It's tempting to simply set the content and display sizes to the same dimensions, as in:

\begin{lstlisting}[language=HTML]
<canvas style="width:640px;height:360px" width="640" height="360">
</canvas>
\end{lstlisting}

There's nothing wrong with the above approach for simple applications.  It's common, however, to specify dynamic styles such as \code{width:100%}.  Moreover CSS pixels don't necessarily correspond to actual device pixels.  This is especially true on high-resolution displays like Apple's Retina displays, where browsers typically upscale CSS pixels by 2x. If you assume that device pixels are 1:1 with CSS pixels, you might see ``blurriness'' in your WebGL canvas due to the upscaling.

To get around these issues, \index{\code{GIZA.init}} (Listing~BLAH) examines the \code{devicePixelRatio} window property to check how much (if any) upscaling the browser is performing.  It also examines the \index{\code{clientWidth}} and \index{\code{clientHeight}} properties of the canvas element to obtain the finalized display area.

\begin{lstlisting}[language=JavaScript]
GIZA.init = function(canvasElement) {

  // Find a canvas element if one wasn't specified.
  var canvas = canvasElement;
  if (!canvas) {
    canvas = document.getElementsByTagName('canvas')[0];
  }

  // Obtain the browser's upscale amount, assuming 1 if unavailable.
  var pixelScale = window.devicePixelRatio || 1;

  // Define a function that adjusts content size.
  var adjustSize = function() {
    var displayWidth = canvas.clientWidth;
    var displayHeight = canvas.clientHeight;
    canvas.width = displayWidth * pixelScale;
    canvas.height = displayHeight * pixelScale;
  };

  adjustSize();
  window.onresize = adjustSize;
\end{lstlisting}

In Listing~BLAH, we adjust the canvas width and height not only during initialization, but also during the window's \index{\code{onresize}} event.

MARGIN NOTE
\begin{comment}
It's also common to adjust the WebGL \emph{viewport} and \emph{projection matrix} during a resize event; more on this in future chapters.
\end{comment}

\subsection{Getting a Context}

The entire WebGL API is exposed through a drawing \index{context} object of type \index{\code{WebGLRenderingContext}}.  This object is obtained by calling \index{\code{getContext}} on a canvas element, like so:

\begin{lstlisting}[language=JavaScript]
  gl = canvas.getContext('experimental-webgl', {antialias: true});
\end{lstlisting}

At the time of this writing, ``\index{experimental-webgl}'' is the only string that can be passed to \code{getContext} for WebGL.  (For the 2D canvas API, the string ``2d'' is used.)

The second argument is a set of optional attributes, as specified in Table~\ref{tab:ContextAttributes}.

\begin{table}[htb]\centering
  \begin{tabular}{ll}
    \hline
    Key & Default Value & Description \\
    \hline
    alpha & true & Alpha channel \\
    depth & true & Depth buffer \\
    stencil & false & Stencil buffer \\
    antialias & true & Enables multisampling (or supersampling) \\
    premultipliedAlpha & true & Specifies compositing behavior with the web page; ignored if alpha is false \\
    preserveDrawingBuffer & false & Automatically resets the draw buffer, independently of \code{gl.clear} \\
    \hline
  \end{tabular}
  \caption{WebGL Context Options.}
  \label{tab:ContextAttributes}
\end{table}

We'll examine these attributes in Table~\ref{tab:ContextAttributes} in greater detail in the next section.

\notetoself{Add references to future sections in the book that deal with context loss and image capture.}

\subsection{Clearing the Canvas}

The only WebGL functions we're discussing in this chapter are the first two methods listed in Table~\ref{tab:Clearing}.

\begin{table}[htb]\centering
  \begin{tabular}{ll}
    \hline
    Method & Argument Types \\
    \hline
    clear(mask) & logical or of values in \Table~\ref{tab:ClearBits} \\
    clearColor(red, green, blue, alpha)  & floating point numbers in [0,1] \\
    clearDepth(value) & floating point number in [0,1] \\
    clearStencil(mask) & integer \\
    \hline
  \end{tabular}
  \caption{WebGL Clear Methods.}
  \label{tab:Clearing}
\end{table}

The \code{clear} command has an immediate\footnote{Not quite!  The web browser and graphics driver can buffer up WebGL commands and execute them later.} effect on the canvas, while \code{clearColor}, \code{clearDepth}, and \code{clearStencil} are simply configuring the WebGL state machine.  Most of the WebGL API consists of state-setting commands; there are only three methods in the entire API that can draw pixels into your canvas:

\begin{itemize}
\item clear
\item drawArrays
\item drawElements
\end{itemize}

We'll learn more about \code{drawArrays} and \code{drawElements} in the next chapter.  Listing~foo shows a typical usage pattern for the \code{clear} method.

\begin{lstlisting}[language=JavaScript]
gl = canvas.getContext('experimental-webgl', {antialias: true});
gl.clearColor(1, 1, 0, 1);
gl.clear(gl.COLOR_BUFFER_BIT);
\end{lstlisting}

The \code{COLOR_BUFFER_BIT} flag is one of the constants that can be combined with a logical ``or'' to specify which drawing layers to include in the canvas.  To reduce the memory footprint, choose the fewest number of required flags.  See tab:ClearBit.  Don't worry about the depth and stencil layers; we'll learn more about them in future chapters.

\begin{table}[htb]\centering
  \begin{tabular}{ll}
    \hline
    Property & Value & Default Value \\
    \hline
    DEPTH\_BUFFER\_BIT   & 0x0100 & 0.0 \\
    STENCIL\_BUFFER\_BIT & 0x0400 & 0x00000000\\
    COLOR\_BUFFER\_BIT   & 0x4000 & (0, 0, 0, 0) \\
    \hline
  \end{tabular}
  \caption{WebGL Clear Bits.}
  \label{tab:ClearBit}
\end{table}

You can find a \code{clear} call in many WebGL code samples on the web, but keep in mind that it's not always required.  Some applications never bother filling the background with a solid color; for example, consider an ``infinite tunnel'' game, as depicted on the far left in Figure~foo.  Since every pixel in the canvas is affected by 3D drawing commands, there's no need to clear the color buffer.  If, however, the tunnel were finite (middle in Figure~foo), there would be an area of the screen that never gets drawn to.  If you don't explicitly perform a clear before doing any drawing, this unpainted region might contain junk (right in Figure~foo).  This is because the pixels in canvas are simply a region of memory; if don't explicitly the clear memory, it'll contain retain whatever junky data happened to be sitting around.  This brings us to \code{preserveDrawingBuffer}.

PRESERVEDRAWINGBUFFER

\subsection{Alpha Compositing}

If you're reading this book, you're probably already familiar with the concept of an alpha channel, invented in the late 70's by Alvy Ray Smith, one of the co-founders of the animation studio where I work.

Briefly, alpha is a floating-point value in the [0,1] interval, treated much like the red, green, and blue components of pixel color.  Alpha can be thought of as the inverse of opacity, although WebGL can interpret it in many ways, as we'll learn in Chapter \notetoself{CHAPTER}.  For now we'll focus on the existence of the alpha channel in the canvas and how it interacts with the HTML page compositor in your browser.

PAGECOMPOSITOR

\begin{comment}
images of a web page that has a background image (Egypt!)
each canvas should have an opaque perspective cube

0,0.25,.5,.5 -- no alpha, css-opacity=1
0,0.25,.5,.5 -- alpha without premultiplied, css-opacity=1
0,0.25,.5,.5 -- alpha with premultiplied, css-opacity=1

0,0.25,.5,.5 -- no alpha, css-opacity=.5
0,0.25,.5,.5 -- alpha without premultiplied, css-opacity=.5
0,0.25,.5,.5 -- alpha with premultiplied, css-opacity=.5
\end{comment}

MARGIN NOTE
\begin{comment}
In this book, we never author children of canvas, but there's nothing wrong with doing so.
\end{comment}

...

%%%%%%%%%%%%%%%%%%%%%%%%%%
\section{Animation Timing}
\summary{How to periodically trigger draw events.}

\begin{lstlisting}[language=JavaScript]
  window.requestAnimationFrame = window.requestAnimationFrame ||
    window.mozRequestAnimationFrame ||
    window.webkitRequestAnimationFrame ||
    window.msRequestAnimationFrame;
\end{lstlisting}

%%%%%%%%%%%%%%%%%%%%%%%%%%%%%%%%
\rrecipe{Recipe 1: Strobe Light}
\summary{The simplest possible WebGL application; animates a solid color with \code{clear}.}

