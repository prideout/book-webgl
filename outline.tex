\thispagestyle{empty}
\label{Proposal}
\LARGE
\noindent WebGL Techniques:

\large
\noindent Building 3D Applications for the Web
\small

\vspace{0.25in}
\noindent book proposal for Taylor \& Francis Group

\noindent \href{mailto:philiprideout@gmail.com}{philiprideout@gmail.com}
\normalsize

\renewcommand{\labelenumi}{Chapter \arabic{enumi}. }

\section*{Summary}
Unique in the market by providing a focus on interactivity, \emph{WebGL Techniques} guides readers through an essential set of rendering techniques and 3D interaction techniques, showcasing a set of small-but-complete web apps (\emph{recipes}) at the end of each chapter.

\section*{About the Author}

Philip Rideout has worked in the field of real-time graphics for over ten years, having played roles at several pioneering graphics companies, including Intergraph, NVIDIA, and Pixar.  He is the sole author of \emph{iPhone 3D Programming} (O'Reilly Media), and a contributing author of \emph{GPU Pro 2} (A K Peters) and \emph{OpenGL Insights} (CRC Press).

\section*{Tentative Outline}

In the following outline, note that some section headings are colored in \textbf{\textcolor{Maroon}{maroon}}; these are the designated \emph{recipes}, which are tutorial-style explanations of complete example programs.

These recipes are examples of \emph{literate programming}, a term coined by Donald Knuth.  His idea was to present complete programs in a book, using ordinary English intermingled with small fragments of code.  Tools can extract the code fragments from the book's source (in this case, \LaTeX), creating complete, ready-to-build projects.

\newcommand{\verbiage}[2] {\item \textbf{#1} \footnotesize#2\normalsize}
\newcommand{\rrecipe}[2] {\item \textbf{\textcolor{Maroon}{#1}} \footnotesize#2\normalsize}
\newcommand{\irecipe}[2] {\item \textbf{\textcolor{Maroon}{#1}} \footnotesize#2\normalsize}
\newcommand{\arecipe}[2] {\item \textbf{\textcolor{Maroon}{#1}} \footnotesize#2\normalsize}

\renewcommand{\labelenumii}{\roman{enumii}. }
\setcounter{enumi}{0}

\begin{enumerate}

\item Preliminaries
\begin{enumerate}
\verbiage{A Brief History of *GL}{Describes WebGL's motivation and ancestry: OpenGL, OpenGL ES, etc}
\verbiage{Building Giza: Literate Programming}{Explains our coding conventions and the \texttt{Giza} library that is built up over the course of the book}
\verbiage{The Assembly Line Metaphor}{High-level overview of the WebGL rendering pipeline}
\verbiage{The Canvas Element}{Explains the \texttt{width} and \texttt{height} attributes, how to handle retina displays}
\verbiage{Animation Timing}{How to periodically trigger a draw events}
\rrecipe{Recipe 1: Strobe Light}{The simplest possible WebGL application; animates a solid color with\texttt{clear}}
\{enumerate}

\item Vertex Shading and Transforms
\begin{enumerate}
\verbiage{Vector Algebra with Javascript}{}
\verbiage{Life of a Vertex}{Describes model, view, and projection transforms}
\verbiage{Shading Language Basics}{Writing \texttt{main} for vertex shaders}
\verbiage{Line Drawing}{Explains the LINES primitive and DrawElements}
\verbiage{Vertex Attributes}{}
\verbiage{Vertex Buffer Objects}{}
\verbiage{Javascript Typed Arrays}{}
\rrecipe{Recipe 2: Color Graph}{Animated wireframe graph of the sinc function; illustrates performance differences between Javascript-side computation and shader-side computation}
\end{enumerate}

\item Downloading and Applying Artistic Content
\begin{enumerate}
\verbiage{Texture Coordinates}{Texture coordinates and wrap modes; as an example, shows how to apply a section of the Mona Lisa to a quad.}
\verbiage{Texture Filtering}{Overview of minification, magnification, and mipmapping; uses small pixel art to illustrate the effects of filtering.}
                  % space invaders!
                  % or, http://opengameart.org/content/japanese-buildings-isometric-strategy-game-mockup
\verbiage{Parametric Surfaces}{When the artist is the computer: procedural generation of simple geometry}
\verbiage{Loading Mesh Data with XMLHttpRequest}{Implements Giza's download functionality; also mentions \texttt{webgl-loader}}
\rrecipe{Recipe 3: Planet Mars}{Downloads a sphere mesh and a mars texture} % Maybe even the two moons?
\end{enumerate}

\item Rendering and Lighting Solid Objects
\begin{enumerate}
\verbiage{Face Orientation}{Explains back-face culling in greater-detail (used in the Planet Mars recipe)}
\verbiage{Dealing with Depth}{Explains the depth buffer, how depth artifacts can arise, etc.}
\verbiage{Lambertian Reflection}{Illustrates a simple ambient-diffuse-specular lighting model}
\rrecipe{Recipe 4: ????}{????} % Taj Mahal?
\end{enumerate}

\item Framebuffer Effects Part I
\begin{enumerate}
\verbiage{Alpha Blending}{}
\verbiage{Framebuffer Objects}{How to create offscreen render targets}
\verbiage{Floating Point Textures}{Explains unclamped colors and how they can be used for general-purpose computation. Shows how additive blending can be used to achieve a ``thickness'' effect.}
\verbiage{Gaussian Blur}{Optimal filtering with minimal taps}
\verbiage{Stencil Buffer}{Brief overview of stencil and how it can be used for CSG and shadows} 
\rrecipe{Recipe 5: A Room with a View}{The interior of a moghul palace; illustrates stenciled fake reflection, a depth-based glass effect and a jaali screen backdrop with HDR Bloom}
\end{enumerate}

\item Framebuffer Effects Part II
\begin{enumerate}
\rrecipe{Cubemap Reflection}{Classic, simple use of cubemaps} % refraction too?
\rrecipe{Bump Mapping}{Art-provided normal maps} % external references: procedural bumping
\rrecipe{Parallax Mapping}{Improved bump mapping}
\rrecipe{Shadow Maps}{Old-school PCF shadow mapping} % Uses PCF to soften edges
\rrecipe{Subsurface Scattering}{Transluscent marble} % similar to my fresnel glass demo
\rrecipe{Motion Blur}{Emphasizing movement with directional blur} % Also refer to the SIGGRAPH 2010 geometric motion blur
\rrecipe{Distance Blur}{Depth of field via post-processing} % Real-time Depth-of-Field (ShaderX 5) -- there also seem to be some dx11 demos for this (ATI's ladybug)
\rrecipe{Ambient Occlusion}{Screen-space ambient occlusion}
\end{enumerate}

\item Animation Techniques
\begin{enumerate}
\rrecipe{Blend Shapes}{Morphs between two objects}
\rrecipe{Bone System}{Bends a humanoid wireframe using mocap data}
\end{enumerate}

\item Interaction Techniques
\begin{enumerate}
\verbiage{Capturing events with jQuery }{}
\irecipe{Trackball}{ Classic mouse-driven rotation } % With momentum and quats!
\irecipe{View Cube}{ Move the camera by clicking hotspots on a cube }
\irecipe{Manipulators}{ Interactive handles for positioning objects }
\irecipe{Selection Buffer}{ Render object IDs into a FBO }
\irecipe{Point Selection}{ Voronoi maps for point-cloud selection }
\end{enumerate}

\item Web Applications
\begin{enumerate}
\verbiage{RequireJS and MVC Frameworks}{}
\arecipe{House Modeler}{Draw quads that can be extruded into buildings}
\arecipe{Set Dresser}{Place objects and spotlights into a scene} % might include uberlight, projection textures, and mirrors
\end{enumerate}

Other Libraries
https://code.google.com/p/webgl-loader/
ThreeJS
osgjs
dancer.js

giza.js --- my library!
I'll use grunt.js


\end{enumerate}
