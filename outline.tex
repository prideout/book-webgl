\thispagestyle{empty}
\label{Proposal}
\LARGE
\noindent WebGL Techniques:

\large
\noindent Building 3D Applications for the Web
\small

\vspace{0.25in}
\noindent book proposal for Taylor & Francis Group

\noindent \href{mailto:philiprideout@gmail.com}{philiprideout@gmail.com}
\normalsize

\section*{Summary}
Unique in the market by providing a focus on interactivity, \emph{WebGL Techniques} guides readers through an essential set of rendering and 3D interaction techniques, showcasing a set of small-but-complete web apps (\emph{recipes}) in each section.

\section*{About the Author}

Philip Rideout has worked in the field of real-time graphics for over ten years, having played roles at several pioneering graphics companies, including Intergraph, NVIDIA, and Pixar.  He is the sole author of \emph{iPhone 3D Programming} (O'Reilly Media), and a contributing author in \emph{GPU Pro 2} (A K Peters) and \emph{OpenGL Insights} (CRC Press).

\section*{Tentative Outline}

\newcounter{recipei}
\setcounter{recipei}{0}
\newcommand{\verbiage}[2] {\item \textbf{\textcolor{commentgreen}{#1}} \footnotesize#2\normalsize}
\newcommand{\rrecipe}[2] {\item \textbf{\textcolor{commentgreen}{#1}} \footnotesize#2\normalsize}
\newcommand{\irecipe}[2] {\item \textbf{\textcolor{commentgreen}{#1}} \footnotesize#2\normalsize}
\newcommand{\arecipe}[2] {\item \textbf{\textcolor{commentgreen}{#1}} \footnotesize#2\normalsize}
\renewcommand{\labelenumi}{}
\begin{enumerate}

\item Preliminaries
\begin{enumerate}[resume]
\verbiage{The Canvas Element and HTML 5}{}
\verbiage{The Life of a Triangle}{}
\verbiage{Vector Algebra with Javascript}{}
\verbiage{Whirlwind tour of GLSL}{}
\end{enumerate}
%
\item Vertex Buffers and Attributes
\begin{enumerate}[resume]
\verbiage{Vertex Buffer Objects}{}
\rrecipe{ParametricSurface}{}
\verbiage{Vertex Attributes}{}
\rrecipe{ColorGraph}{Animated wireframe graph of the sinc function} % Illustrates performance differences
\verbiage{Javascript Typed Arrays}{}
\verbiage{Loading Binaries with XMLHttpRequest}{}
\rrecipe{BuddhaViewer}{}
\end{enumerate}
%
\item Rendering Solids
\begin{enumerate}[resume]
\verbiage{Dealing with Depth}{Explains the depth buffer and how depth artifacts can arise}
\verbiage{Lambertian Reflection}{Illustrates a simple ambient-diffuse-specular lighting model}
\rrecipe{BasicLighting}{Extends the ParametricSurface demo to illustrate lighting and depth}
\rrecipe{CelEffect}{Cartoon rendering with a two-pass silhouette}
\end{enumerate}
%
\item Texture-Based Effects
\begin{enumerate}[resume]
\rrecipe{BumpMapping}{Art-provided normal maps} % external references: procedural bumping
\rrecipe{ParallaxMapping}{Improved bump mapping}
\rrecipe{CubemapReflection}{Classic, simple use of cubemaps} % refraction too?
\end{enumerate}
%
\item Animation Techniques
\begin{enumerate}[resume]
\rrecipe{ParticleSystem}{Fireworks with source-over and additive blending}
\rrecipe{BlendShapes}{Morphs between two objects}
\rrecipe{BoneSystem}{Bends a humanoid wireframe; visualizes CMU mocap data}
\end{enumerate}
%
\item Framebuffer Effects
\begin{enumerate}[resume]
\verbiage{Stencil Buffer}{ Brief overview of stencil and how it can be used for CSG and shadows }
\rrecipe{StenciledReflection}{Fake reflection using stencil}
\verbiage{Framebuffer Objects}{Explains offscreen render targets etc}
\rrecipe{GaussianBlur}{Optimal filtering with minimal taps} % perhaps fold in HDR bloom too
\rrecipe{ShadowMaps}{Old-school PCF shadow mapping} % Uses PCF to soften edges
\rrecipe{SubsurfaceScattering}{Transluscent marble} % similar to my fresnel glass demo
\rrecipe{MotionBlur}{Emphasizing movement with directional blur} % Also refer to the SIGGRAPH 2010 geometric motion blur
\rrecipe{DistanceBlur}{Depth of field via post-processing} % Real-time Depth-of-Field (ShaderX 5) -- there also seem to be some dx11 demos for this (ATI's ladybug)
\rrecipe{AmbientOcclusion}{Screen-space ambient occlusion}
\end{enumerate}
%
\item Interaction Techniques
\begin{enumerate}[resume]
\irecipe{Trackball}{ Classic mouse-driven rotation } % With momentum and quats!
\irecipe{ViewCube}{ Move the camera by clicking on a mini-cube with hotspots }
\irecipe{Manipulators}{ Interactive handles for positioning objects }
\irecipe{SelectionBuffer}{ }
\irecipe{PointSelection}{ Voronoi maps for point-cloud selection }
\end{enumerate}
%
\item Web Applications
\begin{enumerate}[resume]
\arecipe{HouseModeler}{Draw quads that can be extruded into buildings}
\arecipe{SetDresser}{Place objects and spotlights into a scene} % might include uberlight, projection textures, and mirrors
\end{enumerate}

\end{enumerate}
