\documentclass{book}
\usepackage[usenames,dvipsnames]{color}
\usepackage{geometry}
\usepackage{graphics}
\usepackage{graphicx}
\usepackage{listings}
\usepackage{hyperref}
\usepackage{fancyhdr}
\usepackage{mdwlist}
\usepackage{enumitem}
\usepackage{minted}
\usepackage[avantgarde]{quotchap}
\definecolor{DarkKhaki}{rgb}{0.741, 0.718, 0.420} 
\definecolor{chaptergrey}{rgb}{0.741, 0.718, 0.420} 
\definecolor{commentgreen}{RGB}{68, 136, 85}
\usepackage{makeidx}
\usepackage{verbatim}
\usepackage[raggedright,scriptsize]{subfigure}
\usepackage{amsmath, amsthm, amssymb}

\hypersetup {
    colorlinks,
    urlcolor=blue,
    citecolor=magenta
}

\title{WebGL Techniques:\\Building 3D Applications for the Web}
\date{}
\author{Philip Rideout}

\geometry {
  paperwidth=6in,
  paperheight=9in,
  bottom=0.5in,
  top=0.5in,
  marginparsep=0.1in,
  marginpar=0.75in
}

\begin{comment}

Settings > Long Line Settings > 72

   1. \tiny
   2. \scriptsize
   3. \footnotesize
   4. \small
   5. \normalsize
   6. \large
   7. \Large
   8. \LARGE
   9. \huge
  10. \Huge 
\end{comment}

\usemintedstyle{trac}
\definecolor{bg}{rgb}{0.95, 0.95, 0.95}
\newminted{cpp}{gobble=4,bgcolor=bg,texcl,fontsize=\footnotesize}
\newminted{glsl}{gobble=4,bgcolor=bg,texcl,fontsize=\footnotesize}
\newcommand{\greenlinks}{\hypersetup{colorlinks,linkcolor=commentgreen}}
\newcommand{\redlinks}{\hypersetup{colorlinks,linkcolor=red}}
\newcommand{\recipecodefile}[2] { \textsf{#1 \textcolor{red}{\pageref{#2}}}\\ }
\newcommand{\recipemediafile}[1] { \textsf{#1}\\ }
\newcommand{\recipecite}[1] { \textsf{\cite{#1}}\\ }

\newcommand{\newfragment}[2] {
    \phantomsection
    \label{#2}
    \vspace{0.125in}
    \noindent \texttt{\emph{\textbf{\textcolor{commentgreen}{#1}}} $\equiv$ }
%    \nopagebreak
    
}
\newcommand{\addfragment}[2] {
    \phantomsection
    \label{#2}
    \vspace{0.125in}
    \noindent \texttt{\emph{\textbf{\textcolor{commentgreen}{#1}}}  +$\equiv$ }
%    \nopagebreak
    
}

\newcommand{\fragmentref}[2] {\greenlinks \hyperref[#2]{#1} \redlinks >>> \hspace{\fill} \tiny \emph{\pageref{#2} }}

%\newcommand{\inlineref}[2] {\greenlinks\emph{\hyperref[#2]{#1}}\redlinks}
\newcommand{\inlineref}[2] {\greenlinks\hyperref[#2]{#1}\redlinks}

%\usepackage{booklet}
%\source{\magstep0}{6in}{9in}
%\target{\magstep0}{11in}{8.5in}
%\setpdftargetpages
%\pagespersignature{120}

\pagestyle{fancy}

%\usepackage[cam,center,letter,noinfo]{crop}

\graphicspath{{../figs/}}

\makeindex
\begin{document}

\mainmatter
 \renewcommand{\chaptermark}[1]{\markboth{Chapter \thechapter. #1}{}}
 \renewcommand{\sectionmark}[1]{\markright{\thesection. #1}}

% Dedication Page
% \maketitle
% \emph{for Mom, Dad, and Ken}

\thispagestyle{empty}
\label{Proposal}
\LARGE
\noindent WebGL Techniques:

\large
\noindent Building 3D Applications for the Web
\small

\vspace{0.25in}
\noindent book proposal for Taylor \& Francis Group

\noindent \href{mailto:philiprideout@gmail.com}{philiprideout@gmail.com}
\normalsize

\renewcommand{\labelenumi}{Chapter \arabic{enumi}. }

\section*{Summary}
Unique in the market by providing a focus on interactivity, \emph{WebGL Techniques} guides readers through an essential set of rendering techniques and 3D interaction techniques, showcasing a set of small-but-complete web apps (\emph{recipes}) in each chapter.

\section*{About the Author}

Philip Rideout has worked in the field of real-time graphics for over ten years, having played roles at several pioneering graphics companies, including Intergraph, NVIDIA, and Pixar.  He is the sole author of \emph{iPhone 3D Programming} (O'Reilly Media), and a contributing author of \emph{GPU Pro 2} (A K Peters) and \emph{OpenGL Insights} (CRC Press).

\section*{Tentative Outline}

In the following outline, note that some section headings are colored in \textbf{\textcolor{Maroon}{maroon}}; these are the designated \emph{recipes}, which are tutorial-style explanations of complete example programs.

These recipes are examples of \emph{literate programming}, a term coined by Donald Knuth.  His idea was to present complete programs in a book, using ordinary English intermingled with small fragments of code.  Tools can extract the code fragments from the book's source (in this case, \LaTeX), creating complete, ready-to-build projects.

\newcommand{\verbiage}[2] {\item \textbf{#1} \footnotesize#2\normalsize}
\newcommand{\rrecipe}[2] {\item \textbf{\textcolor{Maroon}{#1}} \footnotesize#2\normalsize}
\newcommand{\irecipe}[2] {\item \textbf{\textcolor{Maroon}{#1}} \footnotesize#2\normalsize}
\newcommand{\arecipe}[2] {\item \textbf{\textcolor{Maroon}{#1}} \footnotesize#2\normalsize}

\renewcommand{\labelenumii}{\roman{enumii}. }
\setcounter{enumi}{0}

\begin{enumerate}
\item Preliminaries
\begin{enumerate}
\verbiage{The Canvas Element and HTML 5}{}
\verbiage{The Life of a Triangle}{}
\verbiage{Vector Algebra with Javascript}{}
\verbiage{Whirlwind tour of GLSL}{}
\end{enumerate}
\pagebreak
\item Vertex Buffers and Attributes
\begin{enumerate}
\verbiage{Vertex Buffer Objects}{}
\rrecipe{Parametric Surfaces}{Procedural generation of simple geometry}
\verbiage{Vertex Attributes}{}
\rrecipe{Color Graph}{Animated wireframe graph of the sinc function} % Illustrates performance differences
\verbiage{Javascript Typed Arrays}{}
\verbiage{Loading Binaries with XMLHttpRequest}{}
\rrecipe{Buddha Viewer}{Download \& render a large mesh}
\end{enumerate}
%
\item Rendering Solids
\begin{enumerate}
\verbiage{Dealing with Depth}{} % Explains the depth buffer and how depth artifacts can arise
\verbiage{Lambertian Reflection}{} % Illustrates a simple ambient-diffuse-specular lighting model
\rrecipe{Basic Lighting}{Extends ParametricSurface to illustrate lighting}
\rrecipe{Cel Effect}{Cartoon rendering with a two-pass silhouette}
\end{enumerate}
%
\item Blending and Texture-Based Effects
\begin{enumerate}
\verbiage{Texture Coordinates and Wrap Modes}{}
\verbiage{Filtering and Mipmapping}{}
\verbiage{Alpha Blending}{}
\rrecipe{Bump Mapping}{Art-provided normal maps} % external references: procedural bumping
\rrecipe{Parallax Mapping}{Improved bump mapping}
\rrecipe{Cubemap Reflection}{Classic, simple use of cubemaps} % refraction too?
\rrecipe{Particle System}{Fireworks with additive blending}
\end{enumerate}
%
\item Animation Techniques
\begin{enumerate}
\rrecipe{Blend Shapes}{Morphs between two objects}
\rrecipe{Bone System}{Bends a humanoid wireframe using mocap data}
\end{enumerate}
%
\item Framebuffer Effects
\begin{enumerate}
\verbiage{Stencil Buffer}{} % Brief overview of stencil and how it can be used for CSG and shadows
\rrecipe{Stenciled Reflection}{Fake reflection using stencil}
\verbiage{Framebuffer Objects}{} % Explains offscreen render targets etc
\verbiage{Multiple Render Targets}{} % Explains offscreen render targets etc
\rrecipe{Gaussian Blur}{Optimal filtering with minimal taps} % perhaps fold in HDR bloom too
\rrecipe{Shadow Maps}{Old-school PCF shadow mapping} % Uses PCF to soften edges
\rrecipe{Subsurface Scattering}{Transluscent marble} % similar to my fresnel glass demo
\rrecipe{Motion Blur}{Emphasizing movement with directional blur} % Also refer to the SIGGRAPH 2010 geometric motion blur
\rrecipe{Distance Blur}{Depth of field via post-processing} % Real-time Depth-of-Field (ShaderX 5) -- there also seem to be some dx11 demos for this (ATI's ladybug)
\rrecipe{Ambient Occlusion}{Screen-space ambient occlusion}
\end{enumerate}
%
\item Interaction Techniques
\begin{enumerate}
\verbiage{Capturing events with jQuery }{}
\irecipe{Trackball}{ Classic mouse-driven rotation } % With momentum and quats!
\irecipe{View Cube}{ Move the camera by clicking hotspots on a cube }
\irecipe{Manipulators}{ Interactive handles for positioning objects }
\irecipe{Selection Buffer}{ Render object IDs into a FBO }
\irecipe{Point Selection}{ Voronoi maps for point-cloud selection }
\end{enumerate}
%
\item Web Applications
\begin{enumerate}
\verbiage{RequireJS and MVC Frameworks}{}
\arecipe{House Modeler}{Draw quads that can be extruded into buildings}
\arecipe{Set Dresser}{Place objects and spotlights into a scene} % might include uberlight, projection textures, and mirrors
\end{enumerate}

\end{enumerate}


\def\thesection {Recipe \arabic{section}:}
\def\thesubsection {\arabic{section}.\arabic{subsection}}
\renewcommand{\sectionmark}[1]{\markright{\thesection\ #1}}

%%%%%%%%%%%%%%%%%%%%%%%%%%%%%%%%%%%%%%%%%%%%%%%%%%%%%%%%%%%%%%%%%%%%%%
% Redefine the section command tro be invisible

\makeatletter
\renewcommand\section{\@startsection {section}{1}{\z@}%
                                   {0pt}%
                                   {0pt}%
                                   {\vphantom}}
\makeatother

%%%%%%%%%%%%%%%%%%%%%%%%%%%%%%%%%%%%%%%%%%%%%%%%%%%%%%%%%%%%%%%%%%%%%%
% The "definerecipe" command
% name, files, imagepath, description
\newcommand{\definerecipe}[4]
{
	\vspace{0.1in}
	\marginpar { \raggedright \tiny
		\vspace{0.3in}
		#2
	}
	\section{#1} \ \\
	\addcontentsline{toc}{section}{#1}
	\mbox{ \colorbox{bg}{\begin{minipage}{4in}
	\vspace{0.05in}
	\raggedright \Large \textbf{\thesection\ #1} \normalsize
	\vspace{0.05in}
	\setlength\fboxsep{0pt}
	\begin{tabular}{ l p{2.15in} }
	 \vtop{\vspace{0pt}\hbox{\fbox{\includegraphics[width=1.5in]{../media/screenshots/#3}} }} &
	 \vtop{\vspace{0pt}\parbox{2.15in}{\noindent\small\textsf{#4} } }\\
	\end{tabular}
	\vspace{0.05in}
	\end{minipage} } }
	\vspace{0.2in}
}
%%%%%%%%%%%%%%%%%%%%%%%%%%%%%%%%%%%%%%%%%%%%%%%%%%%%%%%%%%%%%%%%%%%%%%

%\chapter{Preliminaries}

%%%%%%%%%%%%%%%%%%%%%%%%%%%%%%%%
\section{WebGL and its Ancestry}
\summary{Gives an account of WebGL's ancestry (OpenGL, OpenGL ES), motivation, and rapid growth.  Also briefly mentions impediments at the time of writing, such as security concerns and IE support.}

3D graphics in the browser is not new.  One of the first technologies in this area was \index{VRML} VRML, the virtual reality modeling language standardized by the W3C in the mid 90's.  VRML was popular in academia but didn't quite have the same wildfire effect that characterizes the recent explosion of WebGL.

Why has WebGL left other 3D web technologies in the dust?  For one, it came along at just the right time; JavaScript only recently attained status as a serious platform for application development.  Sophisticated tools like Google's V8 \index{V8} engine  and John Resig's jQuery \index{jQuery} library have legitimized JavaScript in the eyes of developers coming from traditional desktop development.

WebGL has plenty of merits that make it more attractive than other browser-based 3D APIs.  Almost all popular browsers support it natively, freeing users from the overhead of plug-ins.  WebGL is also ``close to the metal'' -- by providing low-level access to graphics hardware, developers can maximize performance like never before.

WebGL's ancestry actually lies not in VRML (which did give rise to X3D and other standards) but in a low-level graphics API for C developed by Silicon Graphics in the early 90's.  They initially named their API \emph{IrisGL}, which evolved to \emph{OpenGL} when they released it as an industry standard.   With the rise of mobile platforms, OpenGL spawned OpenGL ES, which encompasses almost the same feature set of WebGL.  In a sense, WebGL is simply a JavaScript binding for OpenGL ES 2.0.

WebGL has another ancestor that is arguably just as influential as IrisGL: the Renderman Shading Language (RSL) developed by Pixar inspired much of the syntax in the language that WebGL provides for authoring \index{shaders} \emph{shaders}, relatively small routines executed with massive parallelism.  Shaders have the final say in the 3D position of every vertex, and the RGB color of every rasterized pixel.  We'll learn more about shaders in Chapter 2.

%%%%%%%%%%%%%%%%%%%%%%%%%%%%%%%%%%%%%%%%%%%%%
\section{Building Giza}
\summary{Describes our coding conventions and the \code{giza} library that is developed over the course of the book.}

This is not just a book, it's a library.  Many of the code listings in the book are lifted directly from the \emph{giza} library, named after the city in ancient Egypt.  The skyline of the Giza Necropolis is composed of triangles, the fundamental drawing primitive in computer graphics.  The pyramid shape also describes a \emph{viewing frustum}, a spatial region that encompasses everything within a certain vantage point in a 3D graphics program.

As a design philosphy, giza never calls methods on the WebGL context object -- that's your job!  We made this rule of thumb to ensure that we cover every detail of the API in the book, and to differentiate giza from higher-level libraries such as ThreeJS.  Instead of providing an abstraction of WebGL, giza provides utilities for performing vector math, constructing buffers of mesh data, and more.

All giza methods belong to a global object called \code{GIZA}, following the namespace convention common in JavaScript.  For example, here's how we define a small utility function that merges all attributes from object \code{b} into object \code{a}:

\begin{lstlisting}[language=JavaScript]
var GIZA = GIZA || {};

GIZA.merge = function (a, b) {
  for (var attrname in b) {
    a[attrname] = b[attrname];
  }
};
\end{lstlisting}

The first line allows clients to include a subset of Giza's source files by creating a \code{GIZA} namespace only if it doesn't already exist.  This isn't necessary when using giza in minified form, since the entire library is included in that case.

Other than \code{GIZA}, the only other global variables that we ever set are \code{COMMON} and \code{gl}.  The \code{gl} variable is the context object that exposes the entire WebGL API, which we'll learn about in Section~\ref{sec:context}.  We made this a global for terseness, since we make a huge number of calls to the context object.

The \code{COMMON} namespace is reserved for a few utility methods used in our recipes that don't belong to giza.  So, we came up with \code{COMMON} as a place to put a small handful of WebGL helpers that can be called from any recipe.  Unlike methods in \code{GIZA}, methods in \code{COMMON} are allowed to call jQuery functions and WebGL methods.

For access to the full source code to giza, or to download the minified library, refer to our github site:

\notetoself{http://to.be.filled.in.com}

Our github project also contains the code for all the recipes in this book, including the HTML and CSS source (and the \code{COMMON} layer).

%%%%%%%%%%%%%%%%%%%%%%%%%%%%
\section{The Canvas Element}
\summary{Explains the \code{width} and \code{height} attributes and how to handle high-dpi displays (e.g., Apple Retina).}

The canvas element \index{\code{<canvas>} element} is one of the cornerstones of the HTML5 platform.  It provides rich drawing APIs for both 2D and 3D graphics.  The 2D API will not be discussed much in this book (except briefly in Chapter 8), and the 3D API is, of course, WebGL.

\subsection{Dealing with Size}

By default, \code{<canvas>} is a block level element, similar to \code{<div>}.  This means that it can only be a child of \code{<body>}, and that it's typically rendered with surrounding line breaks.

An important distinction between canvas and other elements is that it has two \index{size} sizes.  One size is the \emph{display area}, specified with familiar \index{CSS} CSS mechanisms.  The other size is the \emph{content size}, specified with explicit \code{width} \index{\code{width} attribute} and \code{height} \index{\code{height} attribute} attributes.  The content size specifies an off-screen drawing surface that gets scaled into the display area on the web page.

It's tempting to simply set the content and display sizes to the same dimensions, as in:

\begin{lstlisting}[language=HTML]
<canvas style="width:640px; height:360px"
        width="640" height="360">
</canvas>
\end{lstlisting}

There's nothing wrong with the above approach for simple applications.  It's common, however, to specify a dynamicly-sized display area rather than a fixed one (e.g., \code{width:100\%}).  Moreover CSS pixels don't necessarily correspond to actual device pixels.  This is especially true on displays with a high pixel density, where browsers typically upscale CSS pixels by 2x.

\begin{sidenote}
If you assume that device pixels are 1:1 with CSS pixels, you might see blurriness in your WebGL canvas due to upscaling.
\end{sidenote}

To get around these issues, \code{GIZA.init} (Listing~\ref{lst:GIZA:init1}) examines the \code{devicePixelRatio} window property to check how much upscaling is active.  It also examines the \code{clientWidth} \index{\code{clientWidth}, \code{clientHeight}} and \code{clientHeight} properties of the canvas element to obtain the finalized display area.

\begin{lstlisting}[
    caption={Adjusting the Canvas Size},
    label=lst:GIZA:init1,
    language=JavaScript]
GIZA.init = function(canvasElement) {

  // Find a canvas element if one wasn't specified.
  var canvas = canvasElement;
  if (!canvas) {
    canvas = document.getElementsByTagName('canvas')[0];
  }

  // Obtain the browser's upscale amount, assuming 1 if unavailable.
  var pixelScale = window.devicePixelRatio || 1;

  // Define a function that adjusts content size.
  var adjustSize = function() {
    var displayWidth = canvas.clientWidth;
    var displayHeight = canvas.clientHeight;
    canvas.width = displayWidth * pixelScale;
    canvas.height = displayHeight * pixelScale;
  };

  adjustSize();
  window.onresize = adjustSize;
\end{lstlisting} \index{\code{GIZA.init}}

In Listing~\ref{lst:GIZA:init1}, we adjust the canvas width and height not only during initialization, but also in response to the window's \index{\code{onresize} event} \code{onresize} event.

\begin{sidenote}
It's also common to adjust the WebGL \emph{viewport} and \emph{projection matrix} during a resize event.  More on this in future chapters.
\end{sidenote}

\subsection{Getting a Context}
\label{sec:context}

The entire WebGL API is exposed through an object of type \index{\code{WebGLRenderingContext}} \code{WebGLRenderingContext}, commonly known as the \index{draw context} \emph{draw context}.  It's obtained by calling \code{getContext} \index{\code{getContext} method} on a canvas element, like so:

\begin{lstlisting}[language=JavaScript]
  gl = canvas.getContext('experimental-webgl', {antialias: true});
\end{lstlisting}

At the time of this writing, \code{"experimental-webgl"} \index{\code{experimental-webgl}} is the only string that can be passed to \code{getContext} for WebGL.  (For the 2D canvas API, the string \code{"2d"} is used.)

The second argument is a set of optional attributes, as specified in Table~\ref{tab:ContextAttributes}.

\begin{table}[htb]\centering
  \begin{tabular}{lll}
    \hline
    Key & Default & Description \\
    \hline
    alpha & true & Alpha channel \\
    depth & true & Depth buffer \\
    stencil & false & Stencil buffer \\
    antialias & true & Enables multisampling \\
    premultipliedAlpha & true & Specifies compositing behavior; ignored if alpha is false \\
    preserveDrawingBuffer & false & Retains the canvas image from the previous draw cycle \\
    \hline
  \end{tabular}
  \caption{WebGL Context Options.}
  \label{tab:ContextAttributes}
\end{table}

We'll examine the attributes in Table~\ref{tab:ContextAttributes} in greater detail in the next section.

\notetoself{Add references to future sections in the book that deal with context loss and image capture.}

\subsection{Clearing the Canvas}

The only WebGL functions we're discussing in this chapter are the first two methods listed in Table~\ref{tab:Clearing}.

\begin{table}[htb]\centering
  \begin{tabular}{ll}
    \hline
    Method & Argument Types \\
    \hline
    clear(mask) & logical ``or'' of values in Table~\ref{tab:ClearBit} \\
    clearColor(red, green, blue, alpha)  & floating point numbers in [0,1] \\
    clearDepth(value) & floating point number in [0,1] \\
    clearStencil(mask) & integer \\
    \hline
  \end{tabular}
  \caption{WebGL Clear Methods.}
  \label{tab:Clearing}
\end{table}

The \code{clear} command changes the pixels in the canvas, while \code{clearColor}, \code{clearDepth}, and \code{clearStencil} are simply configuring the WebGL state machine.  Most of the WebGL API consists of state-setting commands; in fact there are only three methods in the entire API that can draw pixels into your canvas:

\begin{description}
\item[clear] Fill the color buffer (or depth/stencil).
\item[drawArrays] Draw 3D geometry defined by a sequential list of vertices.
\item[drawElements] Draw 3D geometry defined by a list of indices into a vertex buffer.
\end{description}

We'll learn more about \code{drawArrays} and \code{drawElements} in the next chapter.  Listing~\ref{lst:ClearCanvas} shows a usage example for the \code{clear} method.  It filles the canvas with a solid yellow color.

\begin{lstlisting}[
    caption={Clearing the canvas},
    label=lst:ClearCanvas,
    language=JavaScript]
gl = canvas.getContext('experimental-webgl', {antialias: true});
gl.clearColor(1, 1, 0, 1);
gl.clear(gl.COLOR_BUFFER_BIT);
\end{lstlisting}

The \code{COLOR\_BUFFER\_BIT} flag is one of the constants that can be combined with a logical ``or'' to specify which drawing layers to include in the canvas.  To reduce the memory footprint, choose the fewest number of flags from Table~\ref{tab:ClearBit}.  In fact, don't worry about the depth and stencil layers just yet; we'll learn more about them in future chapters.

\begin{table}[htb]\centering
  \begin{tabular}{lll}
    \hline
    Property & Value & Default Value \\
    \hline
    \code{DEPTH\_BUFFER\_BIT}   & 0x0100 & 0.0 \\
    \code{STENCIL\_BUFFER\_BIT} & 0x0400 & 0x00000000\\
    \code{COLOR\_BUFFER\_BIT}   & 0x4000 & (0, 0, 0, 0) \\
    \hline
  \end{tabular}
  \caption{WebGL Clear Bits.}
  \label{tab:ClearBit}
\end{table}

You can find a \code{clear} call in most of the code samples in this book, but keep in mind that it's not always required.  Some applications never bother filling the background with a solid color.

For example, consider a game inside an infinite tunnel, as depicted on the far left in Figure~\ref{fig:Tunnel}.  Since every pixel in the canvas is affected by 3D drawing commands, there's no need to clear the color buffer.  If, however, the tunnel were finite (middle panel), there would be an area of the screen that never gets drawn to.  If you don't explicitly perform a clear before doing any drawing, the unpainted region might contain a ``dirty'' image (right panel), depending on the \code{preserveDrawingBuffer} option you chose when creating the context.  More on this in the next section.

\begin{figure}[htb]\centering
  \includegraphics[width=120mm]{Tunnels.png}
  \caption{The effects of \code{clear()} and \code{preserveDrawingBuffer}.  The closed tunnel on the far left does not need \code{clear()}.}
  \label{fig:Tunnel}
\end{figure}\index{tunnel}

\subsection{When is the canvas \emph{truly} updated?}
\label{sec:doublebuffer}

You might be familiar with \index{double-buffering} double-buffering if you've used other graphics APIs.  WebGL is always double-buffered.  This means that your drawing commands are actually affecting pixels in a \index{backbuffer} \emph{backbuffer}, which is an offscreen drawing surface.  The browser's render loop presents the finalized backbuffer to the screen in one fell swoop.  This creates the illusion of seamless animation by never allowing users to see a partially complete scene.

One side effect of double-buffering is that that the existing pixels you're overwriting do not necessarily correspond to what was drawn in the previous frame.  Normally this doesn't matter; most applications either clear the existing buffer, or fill it entirely with 3D drawing primitives.

In some situations you want the existing pixels to exactly match the previously-drawn frame.  For example, you might need to read color values from the canvas (the \index{\code{readPixels}} \code{readPixels} method will be discussed in \notetoself{CHAPTER}), or your application might intentionally update only certain regions of the canvas for performance reasons.  This is why \code{preserveDrawingBuffer} exists; it allows you to mimic \index{single-buffering} single-buffering.

You might be thinking: ``Okay, so the canvas isn't immediately updated when I call \code{clear} or \code{drawPixels}, but the \emph{backbuffer} is!''  Nope -- the web browser and the graphics driver are actually buffering up WebGL commands and executing them later.  If you'd like, you can tell WebGL to wait till the buffer is done executing by calling the \index{finish} \code{finish} method.  You'll rarely need this however, so don't use it unless you know you need it.

\subsection{Alpha Compositing}

If you're reading this book, you probably already know about the alpha channel, invented in the late 70's by Alvy Ray Smith, one of the co-founders of Pixar.

Alpha is usually a floating-point value in the [0,1] interval, treated much like the red, green, and blue components of pixel color.  Alpha can be thought of as the inverse of opacity, although WebGL can interpret it in many ways, as we'll learn in \notetoself{CHAPTER}.  For now we'll focus on the existence of the alpha channel in the canvas and how it interacts with the HTML page compositor in your browser.

Consider the case where the canvas is cleared to a half-opaque red color like so:

\begin{lstlisting}[language=JavaScript]
gl.clearColor(1, 0, 0, 0.5);
gl.clear(gl.COLOR_BUFFER_BIT);
\end{lstlisting}

If the \code{premultipledAlpha} option is disabled upon context creation, the RGB values in the canvas are multiplied by alpha before being added into the underlying color, as seen on the right in Figure~\ref{fig:PreMult}.

Another example would be clearing the background to (1, 1, 1, 0.5).  If \code{premultipledAlpha} is enabled in this case, the canvas woud seem completely opaque because adding white to any color yields white.  If \code{premultipledAlpha} were disabled, this would (crudely) brighten the background by averaging it with white.

\begin{figure}[htb]\centering
  \includegraphics[width=120mm]{PreMult.png}
  \caption{Semi-transparent canvas composed over a CSS \code{background-image}.}
  % http://en.wikipedia.org/wiki/File:BYR_color_wheel.svg
  \label{fig:PreMult}
\end{figure}

If the \code{alpha} option is disabled when creating a context, none of this matters -- the \code{premultipledAlpha} option is ignored.  The \code{alpha} option is enabled by default, as is \code{premultipledAlpha}.

\begin{comment}
Note that we're not addressing css-opacity.

In this book, we never add children elements to \code{<canvas>}, but there's nothing wrong with doing so.  On some platforms, this can degrade performance, although this is improving as WebGL implementations are maturing.

http://www.svgopen.org/2005/papers/abstractsvgopen/
http://stackoverflow.com/questions/9491417/when-webgl-decide-to-update-the-display?answertab=votes#tab-top

\end{comment}

%%%%%%%%%%%%%%%%%%%%%%%%%%
\section{Animation Timing}
\summary{How to periodically trigger draw events.}

You wouldn't be learning WebGL unless you were interested in animation.  WebGL applications generally issue all rendering commands for a given 3D scene in a \emph{draw cycle}.  The easiest but also most naive way of doing this in JavaScript is \code{setInterval}:

\begin{lstlisting}[language=JavaScript]
// Call myDrawFunc() every 60th of a second
var delay = 1 / 60;
window.setInterval(myDrawFunc, delay / 1000);
\end{lstlisting}

The above approach isn't ideal for high-performing 3D graphics -- it's preferable to hook in to the browser's native rendering loop.  This allows your animation routine to become idle if the browser has no need to draw itself (e.g., if the window is minimized), saving precious battery life and CPU cycles.  It also aligns animation with the refresh rate of the monitor, preventing chopiness.

At the time of this writing, all major browsers supply a method on \code{window} exactly for this purpose, although it has a vendor-specific prefix until it becomes fully standardized.  Giza's \code{init} function handles this for you, as seen in Listing~\ref{lst:GIZA:init2}

\begin{lstlisting}[
    caption={Initializing \code{requestAnimationFrame}},
    label=lst:GIZA:init2,
    language=JavaScript]
GIZA.init = function(canvasElement) {

  ... // Replace ellipses with Listing~\ref{lst:GIZA:init1}

  window.requestAnimationFrame = window.requestAnimationFrame ||
    window.mozRequestAnimationFrame ||
    window.webkitRequestAnimationFrame ||
    window.msRequestAnimationFrame;
\end{lstlisting} \index{\code{GIZA.init}}

It's important to note that \code{requestAnimationFrame} is actually more like \code{setTimeout} than \code{setInterval} in that you need to call it again if you wish another draw cycle to occur.  We provide a function in the \code{COMMON} namespace that can be called at the end of your function to request another draw cycle, and to check the WebGL error state.  This is good practice to ensure clean, error-free code.  See Listing~\ref{lst:COMMON:endFrame}.

\begin{lstlisting}[
    caption={Common end-of-frame tasks.},
    label=lst:COMMON:endFrame,
    language=JavaScript]
  var err = gl.getError();
  if (err != gl.NO_ERROR) {
    console.error("WebGL error during draw cycle: ", err);
    // Don't request another draw cycle here;
    // this prevents an infinite cascade of errors.
  } else {
    window.requestAnimationFrame(wrappedDrawFunc, GIZA.canvas);
  }
\end{lstlisting} \index{\code{COMMON.endframe}}

%%%%%%%%%%%%%%%%%%%%%%%%%%%%%%%%
\section{Recipe 1: Strobe Light}
\summary{The simplest possible WebGL application; animates a solid color with \code{clear}.}

Every chapter in this book ends with \emph{recipe} that walks through all the code for a certain giza demo.  Each recipe has fairly simple HTML, as seen in Figure~\ref{fig:Recipe1}.

\begin{figure}[htb]\centering
  \includegraphics[width=120mm]{Recipe1.png}
  \caption{Examples of ``recipes'': \code{StrobeLight} from Chapter 1 and \code{BasicLighting} from Chapter 4.}
  \label{fig:Recipe1}
\end{figure}

This chapter's recipe is called \code{StrobeLight}, which isn't too exciting but at least establishes the boilerplate we'll be using for future recipes.  First let's take a look at its HTML; see Listing~\ref{lst:StrobeLight1}.

\begin{lstlisting}[
    caption={\code{StrobeLight.html}},
    label=lst:StrobeLight1,
    language=HTML]
<!DOCTYPE html>
<html lang="en">
  <head>
    <title>StrobeLight</title>
    <link href="css/style.css" rel="stylesheet">
    <script src="lib/head.load.min.js"></script>
    <script src="common.js"></script>
    <script src="StrobeLight.js"></script>
  </head>
  <body>
    <h1>Strobe Light</h1>
    <div class="tagline">
      The only shader-free recipe!
    </div>
    <canvas style="width:640;height:360px">
    </canvas>
    <div id="button-bar">
    </div>
  </body>
</html>
\end{lstlisting} \index{\code{StrobeLight.html}}

While giza has no dependencies on any JavaScript libraries, the COMMON layer does use some libraries to make life a little easier.

\begin{description}
\item[head.js] TODO 
\item[stats.js] TODO
\item[jQuery] TODO
\end{description}

Next up see Listing~\ref{lst:COMMON:file}.  TBD

\begin{lstlisting}[
    caption={\code{common.js}},
    label=lst:COMMON:file,
    language=JavaScript]
// Create a COMMON namespace for a small handful of helper functions
// and constants.
var COMMON = {cdn: "http://ajax.googleapis.com/ajax/libs/"};

// Use HeadJS to load scripts asynchronously, but execute them
// synchronously.
head.js(
  "lib/giza.min.js",
  "lib/stats.min.js",
  COMMON.cdn + "jquery/1.8.0/jquery.min.js",
  COMMON.cdn + "jqueryui/1.9.2/jquery-ui.min.js");

// After all scripts have been loaded AND after the document is
// "Ready", execute the recipe's main() function.
head.ready(main);

... // Replace ellipses with Listing~\ref{lst:COMMON:endFrame}.
\end{lstlisting} \index{\code{common.js}}

Next up see Listing~\ref{lst:StrobeLight2}.  TBD

\begin{lstlisting}[
    caption={\code{StrobeLight.js}},
    label=lst:StrobeLight2,
    language=JavaScript]
var main = function() {

  GIZA.init();

  var draw = function(currentTime) {
    var x = 0.5 + 0.5 * Math.sin(currentTime / 100);
    x = 0.25 + x * 0.5;
    gl.clearColor(x, x, 0, 1);
    gl.clear(gl.COLOR_BUFFER_BIT);
    COMMON.endFrame(draw);
  };

  draw(0);

};
\end{lstlisting} \index{\code{StrobeLight.js}}

TBD.
%\chapter{Vertices and Transforms}

\begin{figure}[htb]\centering
  \includegraphics[width=70mm]{pipeline.jpg}
  \caption{High level view of the WebGL / OpenGL ES pipeline}
  \label{fig:AssemblyLine}
\end{figure}

\section{WebGL at 10,000 feet}
\summary{High-level overview of the WebGL rendering pipeline.}

Earlier we indicated that only two WebGL functions that can actually render 3D geometry: \code{drawArrays} and \code{drawElements}.  Both of these simply push a set of \index{vertices} \term{vertices} into the WebGL pipeline.  The vertices become rasterized into \index{fragments} \term{fragments}, or pixel values in the canvas.  Every vertex is defined by a set of \term{vertex attributes}, one of which is usually an XYZ coordinate.  Vertices come together to form \term{primitives}, which include point sprites, lines, and triangles.

Vertices initially belong to a swath of memory called a \term{vertex buffer}, and WebGL provides a rich API that allows you to specify how heterogenous attributes can be packed into a vertex buffer.  Figure~\ref{fig:AssemblyLine} depicts how vertex data flows through the pipeline, starting off in vertex buffers, eventually transforming into pixels in the canvas.

Two phases of this data flow are programmable: the vertex shader and the fragment shader.  Shaders are bundled together into \index{program object} \term{program objects}.   More precisely, a program object is the linked combination of a compiled vertex shader, a compiled fragment shader, and a set of constant inputs known as \term{uniforms}.

We'll dive deeper into shaders and vertex buffers later in this chapter; for now, see Listing~\ref{lst:Taste} for a taste of the various set-up that must be performed before actually pushing the vertices into the pipeline.

\begin{lstlisting}[
    caption={The typical state-setting that occurs before \code{drawArrays}.},
    escapechar=\%,
    label=lst:Taste,
    language=JavaScript]
// Bind a program object and set up a couple uniforms.
gl.useProgram(program);
gl.uniform1f(program, shininess, 0.5);
gl.uniform4fv(program, color, [1,1,0,1]);

// Bind a vertex buffer and set up the position attribute.
gl.bindBuffer(gl.ARRAY_BUFFER, buffer);
gl.enableVertexAttribArray(attribs.POSITION);
gl.vertexAttribPointer(attribs.POSITION, 3, gl.FLOAT, false, 0, 0);

// Finally, draw two triangles (six vertices), starting at vertex 0.
gl.drawArrays(gl.TRIANGLES, 0, 6);
\end{lstlisting}

\section{Vector Algebra with JavaScript}
\notetoself{This sections walks through giza's implementation of vector and matrix classes.  Rotation, translation, and scale are given a very brief treatment.}

\section{The Typical Life of a Vertex}
\notetoself{Describes model, view, and projection transforms.}

\section{Shading Language Basics}
\notetoself{Explains uniforms, attributes, and varyings.  Walks through trivial \texttt{main} functions for vertex and fragment shaders.}

\section{Program Objects}

\subsection{Fetching and Compiling Shader Strings}

Shaders are specified with multi-line strings that get passed to WebGL for compilation.  There are several ways to embed a multi-line string in your JavaScript code: you can end each line with a backslash, or concatenate each line with the \code{+} operator, or you can create an array of strings that can you glue together with \code{join}.

None of embedded options are ideal, so another idea is to download the entire shader from an external file, which jQuery's \code{get} method makes easy.

In this book's sample code, we've decided to use each recipe's HTML file as a container for all shader strings.  There's plenty of precendent for using \code{<script>} tags to hold things that aren't scripts.  For example, this is common when using template systems such as handlebars (\url{http://handlebarsjs.com}). 

Listing~\ref{lst:embed} shows how we embed a fragment shader in the HTML.

\begin{lstlisting}[
    language=HTML,
    caption={Embedding a shader string in HTML.},
    label=lst:embed]
<script id="simplefs" type="x-shader/x-fragment">
  precision highp float;
  varying vec3 vColor;
  void main() {
      gl_FragColor = vec4(vColor, 1.0);
  }
</script>
\end{lstlisting}

Each shader (or ``snippet'' of a shader, as we'll see later) is given a unique identifier in its enclosing script tag.  From JavaScript, this string can easily be retrieved using jQuery:

\begin{lstlisting}[language=JavaScript]
var fsText = $('#simplefs').text();
\end{lstlisting} % $

Since Giza avoids having a dependency on jQuery, it does something similar using the raw DOM element:

\begin{lstlisting}[language=JavaScript]
var fsText = document.getElementById('simplefs').innerHTML;
\end{lstlisting}

Now that we have a shader string, the next step...

\subsection{Creating Program Objects}

Recall that a \term{program object} program object is the linked combination of a compiled vertex shader, a compiled fragment shader, and a set of uniforms.

\subsection{Setting Uniforms}

\subsection{Giza's \code{compile} function}
\notetoself{Describes a useful abstraction of the WebGL program object.}

\section{Lines and Triangles}
\notetoself{Explains the various primitive types (e.g., \texttt{LINES}), \texttt{drawArrays}, and \texttt{drawElements}.}

\section{Typed Arrays, Vertex Attributes and VBOs}
\notetoself{Shows how heterogeneous data (e.g., colors and positions) can be interleaved and submitted to WebGL.}

\subsection{GIZA.BufferView}

\rrecipe{Recipe 2: Color Wheel}
\notetoself{Spinning wheel with various colors at each vertex.}

%\chapter{Using Art Assets}
\section{Texture Coordinates}
\summary{Quick overview of texture coordinates and wrap modes.  For an example, shows how to apply a section of the Mona Lisa to a quad.}
\section{Texture Filtering}
\summary{Explanation of minification, magnification, and mipmapping; uses small pixel art to illustrate the effects of filtering.} % space invaders! % or, http://opengameart.org/content/japanese-buildings-isometric-strategy-game-mockup
\section{Procedural Art}
\summary{Shows how some assets can be generated on the fly, as opposed to downloaded.} % Generate the geometry with node.js?
\subsection{Parametric Surfaces}
\summary{Using JavaScript to generate basic shapes like spheres and donuts.} % Generate the geometry with node.js?
\subsection{Implicit Paths}
\summary{Using fragment shaders to fill the interior of a bounded 2D region.}
\subsection{Perlin Noise}
\summary{Gives a brief overview of noise and applications; covers some techniques for generating noise in GLSL.} % Javascript too?
\section{Loading Mesh Data with XMLHttpRequest}
\summary{Implements Giza's download functionality; also mentions \texttt{webgl-loader}.}
\rrecipe{Recipe 3: Planet Mars}
\summary{Downloads and renders a sphere mesh and Mars texture; also demonstrates \\\texttt{WEBGL\_compressed\_texture\_s3tc.}} % Maybe even the two moons?

%\chapter{Rendering and Lighting Solid Objects}
\section{Face Orientation}
\summary{Explains polygon winding, \texttt{CULL\_FACE} (used in the Planet Mars recipe) and \texttt{gl\_FrontFacing}.}
\section{Dealing with Depth}
\summary{Explains the depth buffer and how depth artifacts can arise.} % Perhaps drill a hole in the mars geometry as an example
\section{Surface Normals}
\summary{Extends the \emph{Parametric Surfaces} sample to compute surface normals.}
\section{Lambertian Reflection}
\summary{Illustrates a simple ambient-diffuse-specular lighting model.}
\rrecipe{Recipe 4: Cel Shading}
\summary{Applies standard lighting, then snaps the color gradient to a few colors for cartoon-style rendering.} % also possibly add a two-pass silhouette

%\include{backmatter01}
%\include{backmatter02}

\backmatter
 \renewcommand{\chaptermark}[1]{\markboth{#1}{#1}}

\raggedright

%\bibliographystyle{alpha}
%\bibliography{book}
%\printindex

\end{document}
